\chapter{Landscaping Study}

%Replace \lipsum with text.
% You may have as many sections as you please. This is just for reference.

\section {Surveys}
Before building any technology, we must explore the existing technologies of the same doamin, their challenges and drawbacks, accessibility of these technologies to the target people and scalability issues. Many questions came in our mind before we started working on the solutions for information reachability to the community poeple. We  conducted on ground surveys in the information deprived and backward areas of Delhi to find the answers of various questions. What type of problems are generally faced by these people? How people use media and mobile phones for solving their community level grievances? How people gain access to daily information, get their complaints solved, receive benefits of Governemnt schemes? How is government involved in solving these matters? How the grievances are amplified which forces government to solve the problems? We tried to find the answers of these questions by conducting following on ground surveys.

\begin{enumerate}
\item \textbf {Survey of ‘Munirka Village’}
Munirka is an urban area in South West Delhi, located near Jawaharlal Nehru
University (JNU) and Indian Institute of Technology Delhi (IIT Delhi)
Campuses. Munirka is a village where development has started in early 1990’s.
The area is mostly dominated by the jaat community. We entered in a dealer’s shop
and asked about the village life, sarpanch and marginalized group in munirka.

\begin{enumerate}
\item It is now termed as an urban area but the place still follows the hierarchy of
sarpanch, vice-sarpanch and community people.
\item Shop owners were using TV as their source of information. Some youngsters
were listening to the radios for infotainment.
\item People owns the shop and had the information of their residence and their
area.
\item Almost many people owns the mobile phones.
\item Government involvement was very less in their grievances.
\end{enumerate}

\item \textbf {Survey of ‘Vasant Vihar’}
Our next visit was towards vasant vihar. Vasant Vihar is an exclusive
neighborhood located in the South West Delhi district of National Capital Territory
of Delhi. We had a visit to “coolie camp” situated in the same place. It is a slum area
where people were living in adverse conditions. There was no proper sanitation
facilities and no hygienic conditions. More than 3000 jhuggi-jhopdiyas, some pakke
makaans were agglomerated in such a small area. The irony is “Vasant Vihar is one
of the most expensive residential areas in the world” (source – Wikipedia) and it
still has such slum areas.\\
Well, we asked various questions to the residents of that place regarding the
availability of basic amenities. Whether they are able to avail the benefits of various
schemes, are able to solve their problems at community level, or by the involvement
of the MLA of their area “Parmila Tokas”. They said for the issues of water and
electricity availability, they approach to the MLA’s office to put their problems.
Sometimes, Officials or people from some department or ministry come to take
surveys for various statistics related to literacy rate, population count, number of
schools, toiletries etc but problems of local people are not addressed. They take
numbers and put it in records. One of the person standing near the retail shop said
that they make no efforts after the surveys, just take the figures that too improper
and report to higher authorities.

\begin{enumerate}
\item  Women were dependent on male members of their family, were unaware of the community information and were carrying no mobile phones.
\item Youngsters were carrying the smart phones for receiving and making calls
and were using it generally for two applications i.e. Facebook and Whatsapp.
\item Local shop owners of the slum areas were carrying basic phones.
\item People aged between 40 to 60 were using phones mainly for gas booking
purposes and for receiving and making incoming and outgoing calls
respectively.
\item Youngsters wanted job related information.
\item No one was much concerned about health problems and health grievances.
\item People rely on words of their peers. Local people generally got informed from
mouth to mouth communication by their peers.
\item Shop owners are aware of their locality, its problems but have no smart
media and were seemed pre-occupied with their own local problems.
\end{enumerate}

\begin{enumerate}
\item 
\end{enumerate}

\end{enumerate}

\section {Observations and Suggestions}

\begin{enumerate}
\item Political Agenda for all the commissions matters more than the actual solution for the problem.
\item School Association Committee (SMCs), health and sanitation committee (HSC) were not able to solve and address the villagers problem effectively.
\end{enumerate}

\section {Conclusions}
It is analysed from all the surveys that people who live in the underdeveloped areas
like rural areas and the slums do not have a platform where they can get all theinformation about the basic needs which they require in their lifestyle. Information
about administration, education, health schemes, employment, agriculture, gas
cylinder bookings, land acquisitions can be correctly communicated to them which
will help them on fulfilling their needs and eventually develop and making their
lifestyle better. But there are certain problems in the implementation of the
method.

\begin {enumerate}
\item  People do not have even small knowledge of operating the mobile phones
whether basic mobile phones or smart phones. Condition is even worse with
women. Though youth and earning member of the house have phones but
still they do not know more than dialing and receiving the call thus intense
training is required for the proper implementation of the approach.
\item  People are reluctant in sharing the information to the people outside their
community. Or they tell data which is partially or completey false. Thus to
collect data through telephonic surveys, Trust needs to be built that it is for
their welfare only.
\item If started by implementing the approach for all the areas mentioned above, it
is unlikely that it will be implemented properly for all the fields and a large
dataset of information is required and need to be maintained. Thus the
approach should be started by implementing for 1 or 2 areas initially.
\end {enumerate}


\end{itemize}
