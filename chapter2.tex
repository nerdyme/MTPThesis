\chapter{Landscaping Study}

%Replace \lipsum with text.
% You may have as many sections as you please. This is just for reference.
Before building any technology, we must explore the existing technologies of the same doamin, their challenges and drawbacks, accessibility of these technologies to the target people and scalability issues. Many questions came in our mind before we started working on the solutions for information reachability to the community poeple.

\section {Surveys}
We conducted on ground surveys in the information deprived and backward areas of Delhi to find the answers of various questions. What type of problems are generally faced by these people? How people use media and mobile phones for solving their community level grievances? How people gain access to daily information, get their complaints solved, receive benefits of Governemnt schemes? How is government involved in solving these matters? How the grievances are amplified which forces government to solve the problems? We tried to find the answers of these questions by conducting following on ground surveys.

\begin{enumerate}
\item \textbf {Survey of ‘Munirka Village’}
Munirka is an urban area in South West Delhi, located near Jawaharlal Nehru
University (JNU) and Indian Institute of Technology Delhi (IIT Delhi)
Campuses \cite{Munir91:online}. Munirka is a village where development has started in early 1990’s.
The area is mostly dominated by the jaat community. We entered in a dealer’s shop
and asked about the village life, sarpanch and marginalized group in munirka.

Moreover, we found that the sarpanch of munirka himself lives in “Vasant Kunj”
and rarely visits his constituency. That was very disappointing part as he was not
at all involved in solving the problems faced by his people. He also gave us contact
number of vice-sarpanch “Bharat Singh” who lives in a nearby street in munirka.
He said “Munirka is no longer a village and the area was well developed where
everyone owns a smart phone and living a standard life. All shop owners were well
equipped with basic amenities. We also talked to some local shop owners on the
main street. The place was still following the village third-tier Panchayati Raj
System. Some points are summarized below.

\begin{enumerate}
\item It is now termed as an urban area but the place still follows the hierarchy of
sarpanch, vice-sarpanch and community people.
\item Shop owners were using TV as their source of information. Some youngsters
were listening to the radios for infotainment.
\item People owns the shop and had the information of their residence and their
area.
\item Almost many people owns the mobile phones.
\item Government involvement was very less in their grievances.
\end{enumerate}

\item \textbf {Survey of ‘Vasant Vihar’}
Our next visit was towards vasant vihar. Vasant Vihar is an exclusive
neighborhood located in the South West Delhi district of National Capital Territory
of Delhi. We had a visit to “coolie camp” situated in the same place. It is a slum area
where people were living in adverse conditions. There was no proper sanitation
facilities and no hygienic conditions. More than 3000 jhuggi-jhopdiyas, some pakke
makaans were agglomerated in such a small area. The irony is “Vasant Vihar is one
of the most expensive residential areas in the world” (source – Wikipedia) and it
still has such slum areas.\\
Well, we asked various questions to the residents of that place regarding the
availability of basic amenities. Whether they are able to avail the benefits of various
schemes, are able to solve their problems at community level, or by the involvement
of the MLA of their area “Parmila Tokas”. They said for the issues of water and
electricity availability, they approach to the MLA’s office to put their problems.
Sometimes, Officials or people from some department or ministry come to take
surveys for various statistics related to literacy rate, population count, number of
schools, toiletries etc but problems of local people are not addressed. They take
numbers and put it in records. One of the person standing near the retail shop said
that they make no efforts after the surveys, just take the figures that too improper
and report to higher authorities. People were reluctant in answering the questions and were uninterested in sharing
the stories and experiences they face and encounter.

\begin{enumerate}
\item  Women were dependent on male members of their family, were unaware of the community information and were carrying no mobile phones.
\item Youngsters were carrying the smart phones for receiving and making calls
and were using it generally for two applications i.e. Facebook and Whatsapp.
\item Local shop owners of the slum areas were carrying basic phones.
\item People aged between 40 to 60 were using phones mainly for gas booking
purposes and for receiving and making incoming and outgoing calls
respectively.
\item Youngsters wanted job related information.
\item No one was much concerned about health problems and health grievances.
\item People rely on words of their peers. Local people generally got informed from
mouth to mouth communication by their peers.
\item Shop owners are aware of their locality, its problems but have no smart
media and were seemed pre-occupied with their own local problems.
\end{enumerate}


\item \textbf {Survey of Seemapuri}
Seemapuri is mainly a rural zone in Delhi. New Seemapuri is situated at one end of north east Delhi. It has Uttar Pradesh as border on one side and lies adjoining to Dilshad Garden in East Delhi. It is basically a heterogeneous community with multi-cultural, multi-lingual and multi-characteristic features. Most of them earn their bread and butter by picking and sorting of rags. Some are daily wage earners, street vendors, domestic helps, and many other menial jobs which are the main stay of their sustenance. Few of them are also shopkeepers, rickshaw pullers and semi skilled labourers working in the construction sector. The fact remains that many of the families are unable to feed their children with the meagre earnings they make.
\begin{enumerate}
\item There are two slums near dilshad garden metro station, Rajeev camp and Sonia camp which were earlier displaced from some other area of Delhi to Dilshad garden due to ongoing construction.
\item Rag pickers earn their wages by picking and selling waste material which is too less for their livelihood.
\item No government surveillance and cleanliness of the locality is maintained.
\item People are poorly literate and have no source of information to participate in social development and governance related activities.
\end{enumerate}

\end{enumerate}


\section{NGOs}
A non-governmental organization (NGO) is a not-for-profit organization that is independent from states and international governmental organisations. They are usually funded by donations but some avoid formal funding altogether and are run primarily by volunteers. NGOs are highly diverse groups of organizations engaged in a wide range of activities, and take different forms in different parts of the world. We talked to NGO personnel of 'Action India' working in Seemapuri Area for the cause of community peopel and for solving their  community and livelihood problems. 
\begin{enumerate}
\item \textbf {Visit to Action india}
Action India founded in 1976, has taken many big and small path breaking initiatives by grassroots women, which clearly indicates the strong potential in women to become change agents in the process of social transformation. Action India sustains a balance between community based work and the universal struggle for women’s rights. While protesting against wrongs, Action India simultaneously creates alternative modes of self-help, self-esteem and self-assertion.

We had a talk with Mr. Praveen from Action India, project co-ordinator Mr. Deven
and Mr. pramod regarding Action India’s approach in solving problems of women
and various programs like ‘women helping women’, ‘Save the girl child campaign’,
‘Adolescents –Education for equality’, ‘Access to water and sanitation’, ‘Rural
program – looking through a gender less’. Various sanghs like ‘sable mahasangh
maitreyi mela’, ‘Hinsa mukt mahilaye’ , ‘beti utsav’ were started and are being run.

The NGO works for women empowerment program. We have discussed the ways
they follow to solve the problems of women.



 \item \textbf{Cause of Action India}
\begin{enumerate} 
 \item   \textbf{Eliminate Discrimination} : Action India initiated the Mahila Panchayat (women’s courts) as a forum for
dispute resolution and realized need and effectiveness of women’s support groups. With the help of legal resource persons the paralegal workers trained the mahila panchayat members on legal rights of women with a strong focus on gender equality. Paralegal workers from the community, mobilized members for the mahila panchayats and today we have 9 mahila panchayats in Delhi. Mahila Panchayat has 14 paralegals and 225 mahila panchayat members. Mahila Panchayats themselves involve in solving the issues of women and support all the cases without any bias till the end.

 \item \textbf{Facilitate access to education} : For education related issues, they approach to School Management Committees, Dept. of School Education and Literacy, Ministry of Human Resource Development, Government of India which sometimes involve in solving the teacher absenteeism problem, School Mgmt Team, Mid day meal Distribution etc.
 
 \item \textbf{Facilitate access to health care} : They approach to health and sanitation committee where they force them to solve
the issues, go to deliver reports in person, for complaints they seek for constant reports and keep receiving to show in case they don’t entertain. NRHM benefits are also seeked by people.

 \item   \textbf {Enhance access to livelihood and economic rights} : Near the New Seemapuri Road which is approx. at 1 km distant from the Dilshad garden metro station, the complete road is occupied by the rag pickers (kabadi vala)
with their sacks. Due to this, this road smells very much and the residents have
problem with it. But as the Action India volunteer Mr. Praveen told us, this work is
the only livelihood for these rag pickers. Also, the MCD vehicle cleans all the
leftouts by the rag picker daily in the evening. We talked to the rag pickers also
asking the problems faced by them and how do they get it solved. We found from the
conversation that their voice is not heard by anybody nor it is communicated to the
govt. authorities.
When they were told about the android application features, they found the
grievance redressal the most useful. As their primary need is to secure their
livelihood and not the secondary needs which they cannot even think to access.
 \item  \textbf{ Enable participation in governance and development} : They encourage people to participate in the local governance related and development issues of society.
\end{enumerate}

\end{enumerate}




\section {Observations and Suggestions}

\begin{enumerate}
\item Political Agenda for all the commissions matters more than the actual solution for the problem.
\item School Association Committee (SMCs), health and sanitation committee (HSC) were not able to solve and address the villagers problem effectively.
\item Recognizing and delivering on-ground training to human Access Points.
\item Unawareness of HAPs regarding local community problems
\item Data Security and Multi-lingual Support
\item On –ground training is mandatory before launching any scheme, giving any
benefits, introducing ICTD media among people, deploying any technology.
\item Manual intervention and involvements are the key elements in introducing
big changes and turning heads of the people.
\item The local knowledge of village is very important prior introducing any new
model in that place.
Necessity of responsible people in various regulatory authorities, commission
departments, panchayats, Government officers, NGO’s workers, ASHA
workers, school teachers.
\item People should themselves come forward to seek solutions and seek
information and registering complaints.
\item Mobile phones users are many and they can be given on ground training for
making the human access points and local villagers known with the problem
and the technologies.
\end{enumerate}

\section {Conclusions}
It is analysed from all the surveys that people who live in the underdeveloped areas
like rural areas and the slums do not have a platform where they can get all theinformation about the basic needs which they require in their lifestyle. Information
about administration, education, health schemes, employment, agriculture, gas
cylinder bookings, land acquisitions can be correctly communicated to them which
will help them on fulfilling their needs and eventually develop and making their
lifestyle better. But there are certain problems in the implementation of the
method.

\begin {enumerate}
\item  People do not have even small knowledge of operating the mobile phones
whether basic mobile phones or smart phones. Condition is even worse with
women. Though youth and earning member of the house have phones but
still they do not know more than dialing and receiving the call thus intense
training is required for the proper implementation of the approach.
\item  People are reluctant in sharing the information to the people outside their
community. Or they tell data which is partially or completey false. Thus to
collect data through telephonic surveys, Trust needs to be built that it is for
their welfare only.
\item If started by implementing the approach for all the areas mentioned above, it
is unlikely that it will be implemented properly for all the fields and a large
dataset of information is required and need to be maintained. Thus the
approach should be started by implementing for 1 or 2 areas initially.
\end {enumerate}

