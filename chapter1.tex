\chapter{Introduction}

%Replace \lipsum with text.
% You may have as many sections as you please. This is just for reference.
Information and communication technologies for development in decentralization refers to more better information and communication for further development of a society. ICT4D also requires
an understanding of local community development, poverty, agriculture, healthcare,
employment and basic education. This makes ICT4D appropriate technology and if it is
shared openly open source appropriate technology.
Therefore to setup ICTD in a society where people are not equipped with technologies and information communication mediums, there is a need of an application which act as a
platform for the human key points of the society which enables them to spread
information among local people of the society. Decentralization as explained \cite{ict1} helps in easy monitoring of the community issues.

\section{Literature Review and Related Work}
The following papers were reviewed while developing the concept.
\begin{itemize}
\item Emergent Practices Around CGNet Swara, A Voice Forum for Citizen Journalism
in Rural India, ICTD’12 \cite{cgnet} : This paper talks about the initiative of CGNet Swara,
which is a project similar to JMR active in Chhatisgarh. The authors explain the
deployment of the system, and their experiences. It also delves into qualitative
and quantitative analysis of the data coming in, of the callers, topic about which
stories were reported among other things.

\item Designing Mobile Interfaces for Novice and Low-Literacy Users,ACM Transactions
on Computer-Human Interaction. 2011 \cite{design}: This study explores different interfaces
for low-literacy and novice mobile users. The authors conducted two studies com-
paring text-based interfaces to other different alternatives such as, one: automatic
solutions including graphics, spoken dialog and text-free user interfaces and sec-
ond: a live human operator. Based on these studies and interviews conducted
with the subjects, the authors cite results regarding the comfort of novice users
with the different mobile interface components. They also lay down certain design
recommendations while designing mobile user interfaces for such users.

\end{itemize}

\section{On Ground Surveys}
The following surveys are conducted on ground.
\begin{enumerate}
\item \textbf {Survey of Munirka Village} : Munirka is an urban area in South West Delhi, located near Jawaharlal Nehru
University (JNU) and Indian Institute of Technology Delhi (IIT Delhi)
Campuses. Munirka is a village where development has started in early 1990’s.
The area is mostly dominated by the jaat community. We entered in a dealer’s shop
and asked about the village life, sarpanch and marginalized group in munirka. He
himself owned a big shop and carried 2 phones.
\ \\
Moreover, we found that the sarpanch of munirka himself lives in “Vasant Kunj”
and rarely visits his constituency. That was very disappointing part as he was not
at all involved in solving the problems faced by his people. He also gave us contact
number of vice-sarpanch “Bharat Singh” who lives in a nearby street in munirka.
He said “Munirka is no longer a village and the area was well developed where
everyone owns a smart phone and living a standard life. All shop owners were well
equipped with basic amenities. We also talked to some local shop owners on the
main street. The place was still following the village third-tier Panchayati Raj
System but the place can’t follow the support the model of human access points.

\item \textbf{Survey of Vasant Vihar} : Our next visit was towards vasant vihar. Vasant Vihar is an exclusive neighborhood located in the South West Delhi district of National Capital Territory
of Delhi. We had a visit to “coolie camp” situated in the same place. It is a slum area
where people were living in adverse conditions. There was no proper sanitation
facilities and no hygienic conditions. More than 3000 jhuggi-jhopdiyas, some pakke
makaans were agglomerated in such a small area. The irony is “Vasant Vihar is one
of the most expensive residential areas in the world” (source – Wikipedia) and it
still has such slum areas.
\ \\
Well, we asked various questions to the residents of that place regarding the
availability of basic amenities. Whether they are able to avail the benefits of various
schemes, are able to solve their problems at community level, or by the involvement
of the MLA of their area “Parmila Tokas”. They said for the issues of water and
electricity availability, they approach to the MLA’s office to put their problems.
Sometimes, Officials or people from some department or ministry come to take
surveys for various statistics related to literacy rate, population count, number of
schools, toiletries etc but problems of local people are not addressed. They take
numbers and put it in records. One of the person standing near the retail shop said
that they make no efforts after the surveys, just take the figures that too improper
and report to higher authorities.
\ \\
People were reluctant in answering the questions and were uninterested in sharing
the stories and experiences they face and encounter.
\end{enumerate}

\newpage
\section{Survey's Conclusion}
It is analysed from all the surveys that people who live in the underdeveloped areas
like rural areas and the slums do not have a platform where they can get all the
information about the basic needs which they require in their lifestyle. Information
about administration, education, health schemes, employment, agriculture, gas
cylinder bookings, land acquisitions can be correctly communicated to them which
will help them on fulfilling their needs and eventually develop and making their
lifestyle better. But there are certain problems in the implementation of the
method.

\begin{itemize}


\item People do not have even small knowledge of operating the mobile phones
whether basic mobile phones or smart phones. Condition is even worse with
women. Though youth and earning member of the house have phones but
still they do not know more than dialing and receiving the call thus intense
training is required for the proper implementation of the approach.

\item People are reluctant in sharing the information to the people outside their
community. Or they tell data which is partially or completey false. Thus to
collect data through telephonic surveys, Trust needs to be built that it is for
their welfare only.

\item If started by implementing the approach for all the areas mentioned above, it
is unlikely that it will be implemented properly for all the fields and a large
dataset of information is required and need to be maintained. Thus the
approach should be started by implementing for 1 or 2 areas initially.

\end{itemize}



% You may add figures in the following manner.

