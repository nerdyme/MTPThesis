\chapter{NGOs' Web Portal}

%Replace \lipsum with text.
% You may have as many sections as you please. This is just for reference.

\section{Registration}
The registration of the app users is done through the website by the NGO by filling the
required information fields in the registration form. The fields will be the name, age,
gender, occupation, address, district, city, mobile-phone number, residence number,
username (unique) and password.
NGO people will also register themselves through the website and authenticated using
NGO’s Email Address by sending an automated link on mail just after registration. They
will be authenticated as the trusted NGO users.\ref{fig:w15}

\section{Functionalities given to NGO People}

\begin{enumerate}
\item \textbf{Registration of HAPs of a particular village} : ​
They can register the contact
number of the admins of a particular village. Registered admins of the village then
can use the application for achieving the other functionalities. NGOs will train HAPs
of the village by the on-ground training and knowing the feedback from other
localite.\ref{fig:w15}
\ \\
\item \textbf{Launching of Surveys} : ​
NGOs can launch surveys from the website by selecting
from the recent surveys of Gramvaani which will be displayed on the website.
Secondly, NGOs can upload their own surveys to be conducted and they can choose
the village where they want to conduct the survey so that the admins of those
villages will be notified for the surveys through the application. List of surveys will
be sent to the android app user as a notification. Admin can then select the
survey(one at a time) and the contacts/groups from one or more of the three
options described above and he can launch the survey. App server will send the id of
the survey to the GV server along with the contacts. GV server will give the IVR
calls to the contacts and collect the responses which will be the updated to the
admin who launched the survey and on the website for the NGO personnel to view
and use the information of the survey. \ref{fig:w11,fig:w12,fig:w13,fig:w16}
\ \\
\item \textbf{Viewing of Survey results} : ​
Surveys results will be displayed after fetching them
from the gramvaani.
\ \\
\item \textbf{Viewing of details of admin} : ​
NGOs are authenticated to control the access
permissions of the app users.

\end{enumerate}
