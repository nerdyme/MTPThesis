\chapter{Conclusion}

\begin{enumerate}
\item On –ground training is mandatory before launching any scheme, giving any
benefits, introducing ICTD media among people, deploying any technology.
\item Manual intervention and involvements are the key elements in introducing
big changes and turning heads of the people.
\item The local knowledge of village is very important prior introducing any new
model in that place.
\item Necessity of responsible people in various regulatory authorities, commission
departments, panchayats, Government officers, NGO’s workers, ASHA
workers, school teachers.
\item People should themselves come forward to seek solutions and seek
information and registering complaints.
\item Mobile phones users are many and they can be given on ground training for
making the human access points and local villagers known with the problem
and the technologies.
\item Main issues and problems are specific to the villages, to the regions. They
need to be identified and then application can be used and make a great
contribution in actually helping people by various means.
\item For each department, there are separate commissions and agencies working
under them, they can be directly put in link with the problem.

\end{enumerate}